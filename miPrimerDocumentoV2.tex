\documentclass[letterpaper,11pt]{article}

\usepackage[utf8]{inputenc}
\usepackage[T1]{fontenc}
\usepackage[spanish]{babel}
\usepackage{multicol}
\usepackage{enumerate}
\usepackage{slantsc}

\topmargin = -2cm
\oddsidemargin= 0cm
\textheight = 23cm
\textwidth = 17cm

\renewcommand{\shorthandsspanish}{}

\parindent=0mm

\title{Mi primer documento en \LaTeX{}}

\author{Pablo Pérez}

\date{\today{}}

\begin{document}

\thispagestyle{empty}

\maketitle

\tableofcontents

\newpage

\begin{abstract}
Ejemplo de documento \LaTeX{} de la clase {\ttfamily article} con una estructura básica. 
Incluye secciones, subsecciones y una referencia cruzada. También se incluyen varios detalles básicos, el uso de diferentes características de las fuentes, párrafos y listas.
\end{abstract}

\section{Primera sección}\label{primera}

\LaTeX{} nos permite crear distintas clases de documentos, entre las que se pueden mencionar: book, article, report, letter y presentaciones, entre otras.

Todos ellos tienen diferentes características y formato. Por ejemplo, la clase \textit{"book"} (libro) coloca un estilo diferente a las páginas pares e impares, permite crear documentos divididos en capítulos, introducir diferente  tipo de tablas de contenido, entre otros detalles.





En \LaTeX los espacios y saltos de linea no siempre se toman de forma literal.
Por ejemplo, un espacio        es      equivalente a      dos o más. 
Un salto de linea es equivalente a un espacio en blanco.

Dos o más saltos de linea son equivalentes a un salto de linea.

Se pueden introducir saltos de linea mediante la expresión $\backslash\backslash$.

Se puede introducir un salto de página mediante el comando $\backslash$newpage o \textbackslash pagebreak.

\section{Caracteres especiales}

\LaTeX hace uso de varios caracteres para indicar las operaciones que se llevarán a cabo sobre el texto, a estos se les conoce como \emph{caracteres reservados}, todos ellos tienen un significado especial para \LaTeX, por lo que son interpretados por el compilador.

Entre ellos se encuentran:

\textbackslash

\{ y \}

\$

\&

\_

\^{}

\#

\~{}

\%

Para hacer uso de estos caracteres es necesario \textit{escaparlos} o utilizar una expresión que los represente. Por ejemplo:

$A = \{a, b, c\}$

b\textbackslash a $= \frac{a}{b}$


\section{Acentos}

Los acentos en \LaTeX se pueden escribir de la siguiente forma:

\verb@\'a@ = \'a

\verb@\'e@ = \'e

\verb@\'i@ = \'i

\verb@\'o@ = \'o

\verb@\'u@ = \'u

\verb@\'a@ = \'A

\verb@\'e@ = \'E

\verb@\'i@ = \'I

\verb@\'o@ = \'O

\verb@\'u@ = \'U

En el caso de la letra \~n, se puede introducir de esta forma:

\verb@\~n@ = \~n

\verb@\~N@ = \~N

\noindent Otra forma de introducirlos es haciendo uso del paquete \textit{inputenc} el cual permite escribir caracteres en el formato habitual, convirtiéndose internamente el texto introducido a texto LaTeX, de acuerdo con las diferentes tablas de equivalencia para las 
distintas páginas de códigos y de forma completamente transparente al usuario.

\noindent Una manera de llevar a cabo lo anterior es introduciendo al principio de nuestro documento (en el preámbulo, entre \textbackslash documentclass y \textbackslash begin\{document\}) lo siguiente:

\textbackslash usepackage[latin1]\{inputenc\}

si la codificación es iso-8889-1 (también conocida como latin1).

Si deseamos hacer uso de codificación utf-8:

\verb@\usepackage[utf8]{inputenc}@

\noindent Ésta última es la más recomendada por ser un estándar que está siendo adoptado en diversas areas.

\section{Listas}

Podemos introducir una lista básica de la siguiente forma:

\begin{itemize}
\item Primer ítem
\item Segundo ítem
\item Tercer ítem
\end{itemize}

\section{Formato de texto}

Veamos algunas otras maneras de dar formato al texto.

Podemos hacer uso de diferentes tipos de letra, por ejemplo:

\begin{itemize}
\item \textrm{Romana}
% antiguamente se usaba {\rm }

\item \emph{Enfática}
% antiguamente se usaba {\em }

\item \textbf{Negritas}
% antiguamente se usaba {\bf }

\item \textit{itálicas}
% antiguamente se usaba {\it }

\item \textsl{Slanted}
% antiguamente se usaba {\sl }

\item \textsf{Sans Serif}
% antiguamente se usaba {\sf }

\item \textsc{Small Caps}
% antiguamente se usaba {\sc }

\item \texttt{Typewriter}
% antiguamente se usaba {\tt }

\item \underline{Subrayado}

\end{itemize}

Estos tipos de letras se pueden combinar para producir, por ejemplo, letras en estilo versales negritas: \textbf{\textsc{Small Caps}}

\section{Tamaño de letras}

Al igual que con los estilos de la fuente, también podemos especificar un tamaño para las letras, basado en el tamaño de la fuente que utiliza el documento actual.

Una forma de especificar estos tamaños es como se muestra a continuación:

\begin{itemize}

\item {\tiny tiny}
% también se puede utilizar \begin{tiny} ... \end{tiny}

\item {\scriptsize scriptsize}
% también se puede utilizar \begin{scriptsize} ... \end{scriptsize}

\item {\small small}
% también se puede utilizar \begin{small} ... \end{small}

\item {\normalsize normal}

\item {\large large}
% también se puede utilizar \begin{large} ... \end{large}

\item {\Large Large}
% también se puede utilizar \begin{Large} ... \end{Large}

\item {\LARGE LARGE}
% también se puede utilizar \begin{tiny} \end{tiny}

\item {\huge huge}
% también se puede utilizar \begin{huge} ... \end{huge}

\item {\Huge Huge}
% también se puede utilizar \begin{Huge} ... \end{Huge}

\end{itemize}

Todos estos tamaños están basados en el tamaño de la fuente del documento.

Estos tamaños se pueden combinar con los estilos de la sección anterior, por ejemplo:

\medskip

\begin{Large}
\textbf{\textit{Ojalá estuviera poniendo atención.}}
\end{Large}

\section{Texto centrado}

Se puede indicar que una parte del documento aparezca centrado en la caja de texto mediante el siguiente ambiente:

\begin{center}
\LARGE{\textbf{\textit{\LaTeX no es \\ tan complicado!!!}}}
\end{center}

\section{Texto alineado a la derecha}

La siguiente expresión nos permite colocar texto alineado a la derecha:

\hfill Este texto aparecerá alineado al margen derecho.

\bigskip

En este caso solamente alineamos a la derecha \hfill una parte.

Podemos incluir una linea o puntos de la siguiente forma:

\bigskip

Sección 1 \hrulefill Página 17

\bigskip

Sección 2 \dotfill Página 24

\section{Espacios verticales y horizontales}

Los espacios horizontales se pueden introducir mediante la expresión:

\hspace{2in}Dos pulgadas.

\hspace{3cm}Tres centimetros.

\hspace{50pt}Cincuenta puntos.

\hspace{-3cm}En el margen.

$x_{2}$ \hspace{4cm} $x^{2}$

Los espacios verticales se pueden introducir de la siguiente forma:

\vspace{2cm}Texto ubicado dos centimetros abajo.

%\vspace{-5cm} \hspace{6cm} Texto ubicado 5 cm arriba y 6 a la derecha.

%\vspace{3cm}

\section{Uso de columnas}

La expresión \textbf{\textbackslash multicol} nos permite introducir texto en columnas. Para poder hacer uso de esto es necesario cargar el paquete \textbf{multicol}. Veamos un ejemplo:

\begin{multicols}{2}
Este es  un ejemplo de texto en dos columnas, puede notarse que \LaTeX{} se encarga de distrobuir el texto de tal forma que las columnas se muestren de igual tamaño. De la misma forma se puede introducir texto en tres o más columnas.\\
Para el caso de dos columnas se puede hacer uso de \textbf{twocolumn}.\\
También se puede indicar el tamaño de cada una de las columnas, pero para esto es necesario hacer uso del ambiente \textbf{\textbackslash minipage} 
o de \textbf{\textbackslash parbox}.
\end{multicols}

\pagebreak

\section{Notas al pie}

Las notas al pie se pueden introducir mediante la expresión:
\begin{center}
\textbackslash footnote\{ ... \}
\end{center}
Esta expresión se coloca en el lugar en el que aparecerá la etiqueta \footnote{Ejemplo de nota al pie}.

\section{Texto Verbatim}

El entorno \textbf{verbatim} nos permite introducir un bloque de texto en el que se va a respetar la forma del texto y los espacios. Por ejemplo:

\begin{verbatim}
Este texto       debe aparecer   exactamente como lo escribimos.

Se deben respetar los espacios y

saltos de linea. \textbf{negritas} 
\textit{itálicas}
\end{verbatim}

El uso más común de este entorno es para introducir código fuente de algún lenguaje de programación.

\section{Primera sección}

\section{Listas numeradas}

Este tipo de listas se pueden introducir de la siguiente forma:

\begin{enumerate}
	\item Primer paso
	\item Segundo paso
	\item Tercer paso
\end{enumerate}

Se puede indicar un tipo diferente de numeración:

\begin{enumerate}[I]
	\item Primer paso
	\item Segundo paso
	\item Tercer paso
\end{enumerate}

\bigskip

\begin{enumerate}[a]
	\item Primer paso
	\item Segundo paso
	\item Tercer paso
\end{enumerate}

Se puede indicar el tipo de número mediante los caracteres: \textit{a, A, I, i, 1}.

Este tipo de listas también pueden anidarse, por ejemplo:

\begin{enumerate}
	\item Primer paso
		\begin{enumerate}
			\item Verificar primero ..
			\item Verfiricar después ...
			\item Ahora ya podemos continuar.
		\end{enumerate}
	\item Segundo paso
	\item Tercer paso
\end{enumerate}

También podemos indicar el número a usar para iniciar la numeración:

\begin{enumerate}
	\setcounter{enumi}{3}
	\item Primer paso
	\item Segundo paso
	\item Tercer paso
\end{enumerate}

\subsection{Listas no numeradas}

Las listas no numeradas se pueden introducir de la siguiente forma: 

\begin{itemize}
	\item Primer ítem
	\item Segundo ítem
	\item Tercer ítem
\end{itemize}

Podemos modificar el tipo de símbolo usado en un item de la siguiente forma:

\begin{itemize}
	\item Primer ítem
	\item[$\delta$] Segundo ítem
	\item Tercer ítem
\end{itemize}

Si deseamos cambiar el tipo de viñeta usado en los items debemos incluir las siguientes expresiones en el preámbulo.

\begin{verbatim}
\renewcommand{\labelitemi}{$-$}
\renewcommand{\labelitemii}{$\cdot$}
\end{verbatim}

De la misma forma, podemos indicar el tipo de números a usar para las listas numeradas:

\begin{verbatim}
\renewcommand{\labelenumi}{(\Roman{enumi})}
\renewcommand{\labelenumii}{\Roman{enumi}.~\alph{enumii}}
\end{verbatim}

\section{Margenes y tamaño de la caja de texto}

Ésta sección complementa a la seccion \ref{primera} incluyendo ejemplos de tablas escritas en \LaTeX.

Podemos especificar el tamaño de los margenes y de la zona de texto mediante las siguientes expresiones:

\begin{verbatim}
\textheight = medida
\textwidth = medida
\topmargin = medida
\oddsidemargin= medida
\end{verbatim}

Estos tamaños deben especificarse en el preámbulo.

Se debe considerar que \LaTeX{} coloca diferente tamaño de margen para cada clase de documento.

\section{Tabla de contenidos y título}

Se puede introducir una tabla de contenidos de manera automática en nuestro documento mediante la expresión:

\begin{verbatim}
\tableofcontents
\end{verbatim}

En el caso del título, este puede introducirse de la siguiente forma:

\begin{verbatim}
\title{Documento Fuente \LaTeX{}}

\author{Pablo Pérez}

\date{}

\begin{document}

\maketitle
\end{verbatim}

\end{document}
