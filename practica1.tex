\documentclass{article}
\usepackage[utf8]{inputenc}
\usepackage[spanish, mexico]{babel}
\usepackage{amsmath, amsfonts, latexsym}
\title{Crédito}
\author{Raúl Peralta\\
\texttt{raulkik@yahoo.com}}


\begin{document}

\section{Etimologia}
La palabra crédito proviene del latín credititus (sustantivación del verbo credere: creer), que significa "cosa confiada". Así "crédito" en su origen significa entre otras cosas, confiar o tener confianza. Se considerará crédito, el derecho que tiene una persona acreedora a recibir de otra deudora una cantidad en numerario para otros. En general es el cambio de una riqueza presente por una futura, basado en la confianza y solvencia que se concede al deudor. El crédito, según algunos economistas, es una especie de cambio que actúa en el tiempo en vez de actuar en el espacio. Puede ser definido como "el cambio de una riqueza presente por una riqueza futura". Así, si un molinero vende 100 sacos de trigo a un panadero, a 90 días plazo, significa que confía en que llegada la fecha de dicho plazo le será cancelada la deuda. En este caso se dice que la deuda ha sido "a crédito, a plazo". En la vida económica y financiera, se entiende por crédito, por consiguiente, la confianza que se tiene en la capacidad de cumplir, en la posibilidad, voluntad y solvencia de un individuo, por lo que se refiere al cumplimiento de una obligación contraída.

\section{Crédito revolvente}
Los clientes de tarjetas de crédito pueden tener diferentes formas para pagar el uso de su línea de crédito. Por lo general será en cuotas o en modalidad revolving. Los clientes que tienen modalidad revolving pueden realizar un pago menor al total facturado en el período (llamado Pago Mínimo). El saldo (la diferencia entre lo facturado y lo pagado), genera una nueva deuda (revolving) a la que se le aplica la tasa de interés vigente para el período y se adiciona al saldo de deuda de esta modalidad, correspondientes a los períodos anteriores si existieren. Esta deuda puede ser pagada (amortizada) por el cliente de manera diferida en el tiempo.

\end{document}